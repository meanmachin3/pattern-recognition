%%%%%%%%%%%%%%%%%%%%%%%%%%%%%%%%%%%%%%%%%
% Structured General Purpose Assignment
% LaTeX Template
%
% This template has been downloaded from:
% http://www.latextemplates.com
%
% Original author:
%  Ted Pavlic (http://www.tedpavlic.com)
% Modified by:
%  Joe Del Rocco (https://joe.delrocco.org)
%%%%%%%%%%%%%%%%%%%%%%%%%%%%%%%%%%%%%%%%%

%----------------------------------------------------------------------------------------
%  PACKAGES AND CONFIGURATION
%----------------------------------------------------------------------------------------

\documentclass[fleqn]{article}
\usepackage{geometry}
\usepackage{fancyhdr} % For custom headers
\usepackage{lastpage} % To determine the last page for the footer
\usepackage{extramarks} % For headers and footers
\usepackage[most]{tcolorbox} % For problem answer sections
\usepackage{graphicx} % For inserting images
\usepackage{xcolor} % For link coloring
\usepackage[hidelinks]{hyperref} % For URL links (no box or color name)
\usepackage{bm}
\usepackage{amsmath}

% Margins
\geometry{
a4paper,
tmargin=1in,
bmargin=1in,
lmargin=1in,
rmargin=1in,
textwidth=6.5in,
textheight=9.0in,
headsep=0.25in
}

% Header and footer
\pagestyle{fancy}
\lhead{\myName} % Top left header
\chead{\myCourse: \myAssignment} % Top center header
\rhead{\firstxmark} % Top right header
\lfoot{\lastxmark} % Bottom left footer
\cfoot{} % Bottom center footer
\rfoot{Page\ \thepage\ of\ \pageref{LastPage}} % Bottom right footer
\renewcommand\headrulewidth{0.4pt} % Size of the header rule
\renewcommand\footrulewidth{0.4pt} % Size of the footer rule

% Other configurations
\setlength\parindent{0pt} % Removes all indentation from paragraphs
\setlength\parskip{1pt} % Ensures paragraphs are still recognizable as such
\setcounter{secnumdepth}{0} % Removes default section numbers
\setcounter{tocdepth}{3} % Sets depth of table of contents
\linespread{1.1}

% Template values
\newcommand{\myLogo}{starfleet.jpg}
\newcommand{\myName}{Manish Yadav}
\newcommand{\myJobTitle}{3836-6483}
\newcommand{\myCompany}{Starfleet Academy}
\newcommand{\myLocation}{1701 Lincoln Blvd, San Francisco, CA}
\newcommand{\myURL}{www.starfleet.edu}
\newcommand{\myEmail}{m.yadav@ufl.edu}
\newcommand{\myCourse}{EEL\ 6825}
\newcommand{\mySection}{Spring 2020}
\newcommand{\myTeacher}{Dr. Dapeng Wu}
\newcommand{\myAssignment}{Homework 1}
\newcommand{\myDueDate}{Friday,\ February\ 7,\ 2020}
\newcommand{\norm}[1]{\left\lVert#1\right\rVert}


%----------------------------------------------------------------------------------------
%  DOCUMENT STRUCTURE (MACROS & ENVIRONMENTS)
%----------------------------------------------------------------------------------------

% Colored links macro
\newcommand{\hrefcol}[3] {\href{#1}{\textcolor{#3}{#2}}}

% Creates a counter to keep track of the number of problems
\newcounter{homeworkProblemCounter}

% Macro for custom title page signature header
\newsavebox{\myTitleSignature}
\sbox{\myTitleSignature}{%
\begin{tabular*}{\textwidth}{@{}l@{}@{\extracolsep{0.125in}}l@{}}%
\parbox[c][]{2.5in}{{\textbf{\myName} \par}
                    {\small \myJobTitle \par}
                    {\small \hrefcol{mailto:\myEmail}{\myEmail}{blue}} \par}
\end{tabular*}}

% Header and footer for when a page split occurs within a problem environment
\newcommand{\enterProblemHeader}[1]{%
\nobreak\extramarks{#1}{#1 continued on next page\ldots}\nobreak%
\nobreak\extramarks{#1 (continued)}{#1 continued on next page\ldots}\nobreak%
}

% Header and footer for when a page split occurs between problem environments
\newcommand{\exitProblemHeader}[1]{%
\nobreak\extramarks{#1 (continued)}{#1 continued on next page\ldots}\nobreak%
\nobreak\extramarks{#1}{}\nobreak%
}

\newcommand{\homeworkProblemName}{} % Argument = name of problem; default = "Problem #"
\newenvironment{homeworkProblem}[1][Problem \arabic{homeworkProblemCounter}]{%
\stepcounter{homeworkProblemCounter}% % Increase counter for number of problems
\renewcommand{\homeworkProblemName}{#1}% % Assign \homeworkProblemName the argument
\section{\homeworkProblemName}% % Make a section in the document with the custom problem count
\enterProblemHeader{\homeworkProblemName}% % Header and footer within environment
}{%
\exitProblemHeader{\homeworkProblemName}% % Header and footer after environment
}

\newcommand{\problemAnswer}[1]{ % Defines the problem answer command with the content as the only argument
\begin{tcolorbox}[breakable,enhanced,colback=gray!5!white,title=Answer]%
#1
\end{tcolorbox}%
% Alternative - Makes the box around the problem answer and puts the content inside
%\noindent\framebox[\columnwidth][c]{\begin{minipage}{0.98\columnwidth}#1\end{minipage}}
}

\newcommand{\homeworkSectionName}{}
\newenvironment{homeworkSection}[1]{% % For sections w/in problems; Argument = name of section (no default)
\renewcommand{\homeworkSectionName}{#1}% % Assign \homeworkSectionName the argument
\subsection{\homeworkSectionName}% % Make a subsection with the name of the subsection
\enterProblemHeader{\homeworkProblemName\ [\homeworkSectionName]}% % Header and footer within environment
}{%
\enterProblemHeader{\homeworkProblemName}% % Header and footer after environment
}

%----------------------------------------------------------------------------------------
%   TITLE PAGE
%----------------------------------------------------------------------------------------
\begin{document}

% Blank out the traditional title page
\title{\vspace{-1in}} % no title name
\author{} % no author name
\date{} % no date listed
\maketitle % makes this a title page

% Use custom title macro instead
\usebox{\myTitleSignature}
\vspace{1in} % spacing below title header

% Assignment title
{\centering \huge \myAssignment \par}
{\centering \noindent\rule{4in}{0.1pt} \par}
\vspace{0.05in}
{\centering \myCourse~: \mySection~: \myTeacher \par}
{\centering Due \myDueDate \par}
%{\centering Prepared w/ \LaTeX \par}
\vspace{1in}

% Table of Contents
\tableofcontents
\newpage

%----------------------------------------------------------------------------------------
%	PROBLEM 1
%----------------------------------------------------------------------------------------

%\begin{homeworkProblem}[Exercise \#\arabic{homeworkProblemCounter}] % Use for custom section title
\begin{homeworkProblem}
\problemAnswer{
We are given the following equation:
\begin{align*}
g_i(\textsc{x}) &= - \frac{1}{2}\left( \textsc{x} - \mu_i\right)^T\sum^{-1}\left(\textsc{x} - \mu_i\right) + \ln\bm{P}(\omega_i)
\end{align*}
Upon expanding the equation we would get the following,
\begin{align*}
g_i(\textsc{x}) &= - \frac{1}{2}\left( \textsc{x}^T - \mu_i^T\right)\sum^{-1}\left(\textsc{x} - \mu_i\right) + \ln\bm{P}(\omega_i)\\
g_i(\textsc{x}) &= - \frac{1}{2}\left( \textsc{x}^T\sum^{-1} - \mu_i^T\sum^{-1}\right)\left(\textsc{x} - \mu_i\right) + \ln\bm{P}(\omega_i)\\
g_i(\textsc{x}) &= - \frac{1}{2}\left( \textsc{x}^T\sum^{-1}\textsc{x} - \mu_i^T\sum^{-1}\textsc{x} - \textsc{x}^T\sum^{-1}\mu_i + \mu_i^T\sum^{-1}\mu_i\right) + \ln\bm{P}(\omega_i)
\end{align*}
In the above equation, $\textsc{x}^T\sum^{-1}\textsc{x}$ is independent of i and could be ignored. The above equation then could be written as follows: 
\begin{align*}
 g_i(\textsc{x}) &= - \frac{1}{2}\left(-2\mu_i^T\sum^{-1}\textsc{x}+ \mu_i^T\sum^{-1}\mu_i\right) + \ln\bm{P}(\omega_i)   \\
 g_i(\textsc{x}) &= \left( \mu_i^T\sum^{-1}\textsc{x} - \frac{1}{2}\mu_i^T\sum^{-1}\mu_i\right) + \ln\bm{P}(\omega_i)
\end{align*}
Therefore, $g_i(\textsc{x}) = \bm{w}_i^T\textsc{x} +  \omega_{i0}$, where $\bm{w}_i = \sum^{-1}\mu_i$ and $w_{i0} = -\frac{1}{2}\mu_i^T\sum^{-1}\mu_i + \ln\bm{P}(\omega_i)$\\
\\
Since the discriminants are linear, the resulting decision boundaries are again hyperplanes. Let $w_i$ and $w_j$ be two classes. Therefore $g_i(\textsc{x}) - g_j(\textsc{x}) = 0$ defines the equation of decision boundary.

As we already know that $g_i(\textsc{x}) = w_i^T\textsc{x} + w_{i0}$
\begin{align*}
g_i(\textsc{x}) - g_j(\textsc{x}) &= 0\\
(w_i - w_j)^T\textsc{x} + w_{i0} - w_{j0} &= 0\\
\frac{1}{\sigma^2}(\mu_i - \mu_j)^T\textsc{x} - \frac{\mu_i^T\mu_i}{2\sigma^2} + \ln\bm{P}(\omega_i) + \frac{\mu_j^T\mu_j}{2\sigma^2} - \ln\bm{P}(w_j) &= 0\\
(\mu_i - \mu_j)^T\left[\textsc{x} - \left\{ \frac{1}{2}(\mu_i + \mu_j) - \frac{\sigma^2}{\norm{\mu_i - \mu_j}^2 }\ln\frac{\bm{P}(\omega_i)}{\bm{P}(\omega_j)}(\mu_i - \mu_j)\right\} \right] &= 0
\end{align*}

\vspace{5pt}
Which is equivalent to 
\begin{align*}
    \textsc{w}^T(\textsc{x} - \textsc{x}_0) &= 0
\end{align*} where 
\begin{align*}
    \textsc{w} = (\mu_i - \mu_j)
\end{align*}
 and 
 \begin{align*}
     \textsc{x}_0 = \frac{1}{2}(\mu_i + \mu_j) - \frac{\sigma^2}{\norm{\mu_i - \mu_j}^2}\ln\frac{\bm{P}(\omega_i)}{\bm{P}(\omega_j)}(\mu_i - \mu_j)
 \end{align*}
}
\end{homeworkProblem}

\end{document}
%----------------------------------------------------------------------------------------
%	DONE
%----------------------------------------------------------------------------------------
